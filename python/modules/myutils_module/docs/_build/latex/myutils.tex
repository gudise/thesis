%% Generated by Sphinx.
\def\sphinxdocclass{report}
\documentclass[letterpaper,10pt,english]{sphinxmanual}
\ifdefined\pdfpxdimen
   \let\sphinxpxdimen\pdfpxdimen\else\newdimen\sphinxpxdimen
\fi \sphinxpxdimen=.75bp\relax
\ifdefined\pdfimageresolution
    \pdfimageresolution= \numexpr \dimexpr1in\relax/\sphinxpxdimen\relax
\fi
%% let collapsible pdf bookmarks panel have high depth per default
\PassOptionsToPackage{bookmarksdepth=5}{hyperref}

\PassOptionsToPackage{booktabs}{sphinx}
\PassOptionsToPackage{colorrows}{sphinx}

\PassOptionsToPackage{warn}{textcomp}
\usepackage[utf8]{inputenc}
\ifdefined\DeclareUnicodeCharacter
% support both utf8 and utf8x syntaxes
  \ifdefined\DeclareUnicodeCharacterAsOptional
    \def\sphinxDUC#1{\DeclareUnicodeCharacter{"#1}}
  \else
    \let\sphinxDUC\DeclareUnicodeCharacter
  \fi
  \sphinxDUC{00A0}{\nobreakspace}
  \sphinxDUC{2500}{\sphinxunichar{2500}}
  \sphinxDUC{2502}{\sphinxunichar{2502}}
  \sphinxDUC{2514}{\sphinxunichar{2514}}
  \sphinxDUC{251C}{\sphinxunichar{251C}}
  \sphinxDUC{2572}{\textbackslash}
\fi
\usepackage{cmap}
\usepackage[T1]{fontenc}
\usepackage{amsmath,amssymb,amstext}
\usepackage{babel}



\usepackage{tgtermes}
\usepackage{tgheros}
\renewcommand{\ttdefault}{txtt}



\usepackage[Bjarne]{fncychap}
\usepackage{sphinx}

\fvset{fontsize=auto}
\usepackage{geometry}


% Include hyperref last.
\usepackage{hyperref}
% Fix anchor placement for figures with captions.
\usepackage{hypcap}% it must be loaded after hyperref.
% Set up styles of URL: it should be placed after hyperref.
\urlstyle{same}

\addto\captionsenglish{\renewcommand{\contentsname}{Contents:}}

\usepackage{sphinxmessages}
\setcounter{tocdepth}{3}
\setcounter{secnumdepth}{3}


\title{myutils}
\date{Dec 14, 2023}
\release{1.0}
\author{Guillermo Díez\sphinxhyphen{}Señorans}
\newcommand{\sphinxlogo}{\vbox{}}
\renewcommand{\releasename}{Release}
\makeindex
\begin{document}

\ifdefined\shorthandoff
  \ifnum\catcode`\=\string=\active\shorthandoff{=}\fi
  \ifnum\catcode`\"=\active\shorthandoff{"}\fi
\fi

\pagestyle{empty}
\sphinxmaketitle
\pagestyle{plain}
\sphinxtableofcontents
\pagestyle{normal}
\phantomsection\label{\detokenize{index::doc}}


\sphinxstepscope


\chapter{myutils package}
\label{\detokenize{myutils:module-myutils}}\label{\detokenize{myutils:myutils-package}}\label{\detokenize{myutils::doc}}\index{module@\spxentry{module}!myutils@\spxentry{myutils}}\index{myutils@\spxentry{myutils}!module@\spxentry{module}}\index{BarraProgreso (class in myutils)@\spxentry{BarraProgreso}\spxextra{class in myutils}}

\begin{fulllineitems}
\phantomsection\label{\detokenize{myutils:myutils.BarraProgreso}}
\pysigstartsignatures
\pysiglinewithargsret{\sphinxbfcode{\sphinxupquote{class\DUrole{w}{ }}}\sphinxcode{\sphinxupquote{myutils.}}\sphinxbfcode{\sphinxupquote{BarraProgreso}}}{\sphinxparam{\DUrole{n}{total}}\sphinxparamcomma \sphinxparam{\DUrole{n}{long\_barra}\DUrole{o}{=}\DUrole{default_value}{40}}\sphinxparamcomma \sphinxparam{\DUrole{n}{caracter}\DUrole{o}{=}\DUrole{default_value}{\textquotesingle{}\#\textquotesingle{}}}}{}
\pysigstopsignatures
\sphinxAtStartPar
Bases: \sphinxcode{\sphinxupquote{object}}

\sphinxAtStartPar
Esta clase instancia un objeto BarraProgreso.
\begin{quote}\begin{description}
\sphinxlineitem{Parameters}\begin{itemize}
\item {} 
\sphinxAtStartPar
\sphinxstyleliteralstrong{\sphinxupquote{total}} (\sphinxstyleliteralemphasis{\sphinxupquote{int}}) \textendash{} Número de iteraciones que hacen el 100\% de la barra.

\item {} 
\sphinxAtStartPar
\sphinxstyleliteralstrong{\sphinxupquote{long\_barra}} (\sphinxstyleliteralemphasis{\sphinxupquote{int}}\sphinxstyleliteralemphasis{\sphinxupquote{, }}\sphinxstyleliteralemphasis{\sphinxupquote{opcional}}) \textendash{} Número de caracteres que componen la barra, por defecto 40.

\item {} 
\sphinxAtStartPar
\sphinxstyleliteralstrong{\sphinxupquote{caracter}} (\sphinxstyleliteralemphasis{\sphinxupquote{char}}\sphinxstyleliteralemphasis{\sphinxupquote{, }}\sphinxstyleliteralemphasis{\sphinxupquote{opcional}}) \textendash{} Caracter que se utiliza para decorar la barra, por defecto ‘\#’.

\end{itemize}

\sphinxlineitem{Return type}
\sphinxAtStartPar
None

\end{description}\end{quote}

\end{fulllineitems}

\index{export\_legend\_plt() (in module myutils)@\spxentry{export\_legend\_plt()}\spxextra{in module myutils}}

\begin{fulllineitems}
\phantomsection\label{\detokenize{myutils:myutils.export_legend_plt}}
\pysigstartsignatures
\pysiglinewithargsret{\sphinxcode{\sphinxupquote{myutils.}}\sphinxbfcode{\sphinxupquote{export\_legend\_plt}}}{\sphinxparam{\DUrole{n}{ax}}\sphinxparamcomma \sphinxparam{\DUrole{n}{name}\DUrole{o}{=}\DUrole{default_value}{\textquotesingle{}legend.pdf\textquotesingle{}}}\sphinxparamcomma \sphinxparam{\DUrole{n}{pad\_inches}\DUrole{o}{=}\DUrole{default_value}{0.1}}\sphinxparamcomma \sphinxparam{\DUrole{o}{**}\DUrole{n}{kwargs}}}{}
\pysigstopsignatures
\sphinxAtStartPar
Exporta la leyenda de la figura actualmente activa.

\sphinxAtStartPar
Esta función es útil para exportar la leyenda de un gráfico cuando se tienen
múltiples gráficos que comparten la misma leyenda. Es importante tener en cuenta
que se requiere haber creado una figura con ‘plt.plot’, la cual debe incluir una
leyenda, y NO se debe haber ejecutado ‘plt.show()’.
\begin{quote}\begin{description}
\sphinxlineitem{Parameters}\begin{itemize}
\item {} 
\sphinxAtStartPar
\sphinxstyleliteralstrong{\sphinxupquote{ax}} (\sphinxtitleref{matplotlib.axes.Axes}) \textendash{} Los ejes de la figura de los cuales se quiere extraer la leyenda. Si se utilizan
comandos como ‘plot’, etc., se puede obtener los ejes activos del gráfico con
ax=plt.gca().

\item {} 
\sphinxAtStartPar
\sphinxstyleliteralstrong{\sphinxupquote{name}} (\sphinxstyleliteralemphasis{\sphinxupquote{str}}\sphinxstyleliteralemphasis{\sphinxupquote{, }}\sphinxstyleliteralemphasis{\sphinxupquote{opcional}}) \textendash{} Nombre de la imagen. Por defecto es “legend.pdf”.

\item {} 
\sphinxAtStartPar
\sphinxstyleliteralstrong{\sphinxupquote{pad\_inches}} (\sphinxstyleliteralemphasis{\sphinxupquote{float}}\sphinxstyleliteralemphasis{\sphinxupquote{, }}\sphinxstyleliteralemphasis{\sphinxupquote{opcional}}) \textendash{} Esta opción se pasa directamente a la función “savefig”. Por defecto es 0.1.

\item {} 
\sphinxAtStartPar
\sphinxstyleliteralstrong{\sphinxupquote{kwargs}} (\sphinxstyleliteralemphasis{\sphinxupquote{dict}}) \textendash{} Las opciones introducidas aquí se pasan directamente a la función \sphinxtitleref{ax.legend}.

\end{itemize}

\sphinxlineitem{Return type}
\sphinxAtStartPar
None

\end{description}\end{quote}

\end{fulllineitems}

\index{run\_in\_parallel() (in module myutils)@\spxentry{run\_in\_parallel()}\spxextra{in module myutils}}

\begin{fulllineitems}
\phantomsection\label{\detokenize{myutils:myutils.run_in_parallel}}
\pysigstartsignatures
\pysiglinewithargsret{\sphinxcode{\sphinxupquote{myutils.}}\sphinxbfcode{\sphinxupquote{run\_in\_parallel}}}{\sphinxparam{\DUrole{n}{func}}\sphinxparamcomma \sphinxparam{\DUrole{n}{args}}}{}
\pysigstopsignatures
\sphinxAtStartPar
Ejecuta la función en paralelo utilizando argumentos proporcionados.

\sphinxAtStartPar
Esta función toma un método ‘func’ y una lista de argumentos ‘args’, y repite la ejecución
de la función en paralelo. El método ‘func’ puede recibir un número arbitrario y tipos de argumentos,
pero serán pasados en orden según aparezcan en ‘args’.
\begin{quote}\begin{description}
\sphinxlineitem{Parameters}\begin{itemize}
\item {} 
\sphinxAtStartPar
\sphinxstyleliteralstrong{\sphinxupquote{func}} (\sphinxstyleliteralemphasis{\sphinxupquote{function}}) \textendash{} Procedimiento que puede tomar un número arbitrario y tipos de argumentos, pero son específicos
en su posición.

\item {} 
\sphinxAtStartPar
\sphinxstyleliteralstrong{\sphinxupquote{args}} (\sphinxstyleliteralemphasis{\sphinxupquote{list}}) \textendash{} Lista que contiene una lista con los argumentos a pasar a ‘func’ en cada proceso.

\end{itemize}

\sphinxlineitem{Returns}
\sphinxAtStartPar
Lista con los resultados de cada proceso.

\sphinxlineitem{Return type}
\sphinxAtStartPar
list

\end{description}\end{quote}

\begin{sphinxadmonition}{warning}{Warning:}
\sphinxAtStartPar
Esta función siempre debe ser ejecutada dentro de \sphinxtitleref{if \_\_name\_\_==’\_\_main\_\_’:}.
\end{sphinxadmonition}

\end{fulllineitems}

\index{set\_size\_plt() (in module myutils)@\spxentry{set\_size\_plt()}\spxextra{in module myutils}}

\begin{fulllineitems}
\phantomsection\label{\detokenize{myutils:myutils.set_size_plt}}
\pysigstartsignatures
\pysiglinewithargsret{\sphinxcode{\sphinxupquote{myutils.}}\sphinxbfcode{\sphinxupquote{set\_size\_plt}}}{\sphinxparam{\DUrole{n}{ax}}\sphinxparamcomma \sphinxparam{\DUrole{n}{x}}\sphinxparamcomma \sphinxparam{\DUrole{n}{y}}}{}
\pysigstopsignatures
\sphinxAtStartPar
Esta función permite dibujar una figura fijando el tamaño del plot, y no de la figura completa. Toma los valores \sphinxtitleref{x}, \sphinxtitleref{y} que representan las dimensiones en dichos ejes de un plot, y construye la figura del tamaño que sea necesario para acomodar los ejes.
\begin{quote}\begin{description}
\sphinxlineitem{Parameters}\begin{itemize}
\item {} 
\sphinxAtStartPar
\sphinxstyleliteralstrong{\sphinxupquote{ax}} (\sphinxtitleref{matplotlib.axes.Axes}) \textendash{} Objeto que contiene la figura a redimensionar.

\item {} 
\sphinxAtStartPar
\sphinxstyleliteralstrong{\sphinxupquote{x}} (\sphinxstyleliteralemphasis{\sphinxupquote{float}}) \textendash{} Nueva anchura.

\item {} 
\sphinxAtStartPar
\sphinxstyleliteralstrong{\sphinxupquote{y}} (\sphinxstyleliteralemphasis{\sphinxupquote{float}}) \textendash{} Nueva altura.

\end{itemize}

\sphinxlineitem{Return type}
\sphinxAtStartPar
None

\end{description}\end{quote}

\end{fulllineitems}



\section{Submodules}
\label{\detokenize{myutils:submodules}}

\section{myutils.bool module}
\label{\detokenize{myutils:module-myutils.bool}}\label{\detokenize{myutils:myutils-bool-module}}\index{module@\spxentry{module}!myutils.bool@\spxentry{myutils.bool}}\index{myutils.bool@\spxentry{myutils.bool}!module@\spxentry{module}}\index{LinearSystECC (class in myutils.bool)@\spxentry{LinearSystECC}\spxextra{class in myutils.bool}}

\begin{fulllineitems}
\phantomsection\label{\detokenize{myutils:myutils.bool.LinearSystECC}}
\pysigstartsignatures
\pysiglinewithargsret{\sphinxbfcode{\sphinxupquote{class\DUrole{w}{ }}}\sphinxcode{\sphinxupquote{myutils.bool.}}\sphinxbfcode{\sphinxupquote{LinearSystECC}}}{\sphinxparam{\DUrole{n}{dim}\DUrole{o}{=}\DUrole{default_value}{4}}\sphinxparamcomma \sphinxparam{\DUrole{n}{length}\DUrole{o}{=}\DUrole{default_value}{7}}\sphinxparamcomma \sphinxparam{\DUrole{n}{G}\DUrole{o}{=}\DUrole{default_value}{False}}\sphinxparamcomma \sphinxparam{\DUrole{n}{H}\DUrole{o}{=}\DUrole{default_value}{False}}}{}
\pysigstopsignatures
\sphinxAtStartPar
Bases: \sphinxcode{\sphinxupquote{object}}

\sphinxAtStartPar
Clase para manipular sistemas de códigos de corrección de errores lineales.

\sphinxAtStartPar
Esta clase proporciona funcionalidades para la generación de matrices G y H, codificación y decodificación de vectores para códigos de corrección de errores lineales en un cuerpo finito.
\begin{quote}\begin{description}
\sphinxlineitem{Parameters}\begin{itemize}
\item {} 
\sphinxAtStartPar
\sphinxstyleliteralstrong{\sphinxupquote{dim}} (\sphinxstyleliteralemphasis{\sphinxupquote{int}}\sphinxstyleliteralemphasis{\sphinxupquote{, }}\sphinxstyleliteralemphasis{\sphinxupquote{opcional}}) \textendash{} Dimensión de la matriz. Por defecto 4.

\item {} 
\sphinxAtStartPar
\sphinxstyleliteralstrong{\sphinxupquote{length}} (\sphinxstyleliteralemphasis{\sphinxupquote{int}}\sphinxstyleliteralemphasis{\sphinxupquote{, }}\sphinxstyleliteralemphasis{\sphinxupquote{opcional}}) \textendash{} Longitud de la matriz. Por defecto 7.

\item {} 
\sphinxAtStartPar
\sphinxstyleliteralstrong{\sphinxupquote{G}} (\sphinxstyleliteralemphasis{\sphinxupquote{list}}\sphinxstyleliteralemphasis{\sphinxupquote{ of }}\sphinxstyleliteralemphasis{\sphinxupquote{list}}\sphinxstyleliteralemphasis{\sphinxupquote{ of }}\sphinxstyleliteralemphasis{\sphinxupquote{bool}}\sphinxstyleliteralemphasis{\sphinxupquote{, }}\sphinxstyleliteralemphasis{\sphinxupquote{opcional}}) \textendash{} Matriz booleana G para operaciones de generación de código. Por defecto False.

\item {} 
\sphinxAtStartPar
\sphinxstyleliteralstrong{\sphinxupquote{H}} (\sphinxstyleliteralemphasis{\sphinxupquote{list}}\sphinxstyleliteralemphasis{\sphinxupquote{ of }}\sphinxstyleliteralemphasis{\sphinxupquote{list}}\sphinxstyleliteralemphasis{\sphinxupquote{ of }}\sphinxstyleliteralemphasis{\sphinxupquote{bool}}\sphinxstyleliteralemphasis{\sphinxupquote{, }}\sphinxstyleliteralemphasis{\sphinxupquote{opcional}}) \textendash{} Matriz booleana H para operaciones de generación de código. Por defecto False.

\end{itemize}

\sphinxlineitem{Raises}
\sphinxAtStartPar
\sphinxstyleliteralstrong{\sphinxupquote{ValueError}} \textendash{} Si la matriz generadora y la matriz de paridad no cumplen GH\textasciicircum{}=0.

\end{description}\end{quote}
\subsubsection*{Notes}

\sphinxAtStartPar
Esta clase permite realizar operaciones en un cuerpo finito para códigos de corrección de errores lineales.
Los métodos ‘encode’ y ‘decode’ permiten la codificación y decodificación de vectores respectivamente,
usando las matrices generadora y de paridad.
\index{decode() (myutils.bool.LinearSystECC method)@\spxentry{decode()}\spxextra{myutils.bool.LinearSystECC method}}

\begin{fulllineitems}
\phantomsection\label{\detokenize{myutils:myutils.bool.LinearSystECC.decode}}
\pysigstartsignatures
\pysiglinewithargsret{\sphinxbfcode{\sphinxupquote{decode}}}{\sphinxparam{\DUrole{n}{vector}}}{}
\pysigstopsignatures
\sphinxAtStartPar
Decodifica un código de entrada de ‘length’ bits en un mensaje de ‘dim’ bits
utilizando un método de mínima distancia de Hamming.

\sphinxAtStartPar
Este método es absolutamente ineficiente y no debería usarse con dimensiones
mayores que 9.
\begin{quote}\begin{description}
\sphinxlineitem{Parameters}
\sphinxAtStartPar
\sphinxstyleliteralstrong{\sphinxupquote{vector}} (\sphinxstyleliteralemphasis{\sphinxupquote{list}}\sphinxstyleliteralemphasis{\sphinxupquote{ of }}\sphinxstyleliteralemphasis{\sphinxupquote{bool}}) \textendash{} Código booleano de entrada a decodificar.

\sphinxlineitem{Returns}
\sphinxAtStartPar
Mensaje decodificado.

\sphinxlineitem{Return type}
\sphinxAtStartPar
list of bool

\sphinxlineitem{Raises}
\sphinxAtStartPar
\sphinxstyleliteralstrong{\sphinxupquote{ValueError}} \textendash{} Si la longitud del vector de entrada no coincide con ‘length’.
    Si hay demasiados errores para decodificar el vector de manera fiable.

\end{description}\end{quote}

\end{fulllineitems}

\index{encode() (myutils.bool.LinearSystECC method)@\spxentry{encode()}\spxextra{myutils.bool.LinearSystECC method}}

\begin{fulllineitems}
\phantomsection\label{\detokenize{myutils:myutils.bool.LinearSystECC.encode}}
\pysigstartsignatures
\pysiglinewithargsret{\sphinxbfcode{\sphinxupquote{encode}}}{\sphinxparam{\DUrole{n}{vector}}}{}
\pysigstopsignatures
\sphinxAtStartPar
Codifica un mensaje de entrada de ‘dim’ bits en un código de ‘length’ bits.
\begin{quote}\begin{description}
\sphinxlineitem{Parameters}
\sphinxAtStartPar
\sphinxstyleliteralstrong{\sphinxupquote{vector}} (\sphinxstyleliteralemphasis{\sphinxupquote{list}}\sphinxstyleliteralemphasis{\sphinxupquote{ of }}\sphinxstyleliteralemphasis{\sphinxupquote{bool}}) \textendash{} Mensaje booleano de entrada a codificar.

\sphinxlineitem{Returns}
\sphinxAtStartPar
Mensaje booleano codificado.

\sphinxlineitem{Return type}
\sphinxAtStartPar
numpy.ndarray

\sphinxlineitem{Raises}
\sphinxAtStartPar
\sphinxstyleliteralstrong{\sphinxupquote{ValueError}} \textendash{} Si la longitud del vector de entrada no coincide con ‘dim’.

\end{description}\end{quote}

\end{fulllineitems}


\end{fulllineitems}

\index{gauss\_elimination\_gf2() (in module myutils.bool)@\spxentry{gauss\_elimination\_gf2()}\spxextra{in module myutils.bool}}

\begin{fulllineitems}
\phantomsection\label{\detokenize{myutils:myutils.bool.gauss_elimination_gf2}}
\pysigstartsignatures
\pysiglinewithargsret{\sphinxcode{\sphinxupquote{myutils.bool.}}\sphinxbfcode{\sphinxupquote{gauss\_elimination\_gf2}}}{\sphinxparam{\DUrole{n}{coeff\_matrix\_in}}\sphinxparamcomma \sphinxparam{\DUrole{n}{const\_vector\_in}}}{}
\pysigstopsignatures
\sphinxAtStartPar
Resuelve un sistema de ecuaciones booleanas en un cuerpo finito con las operaciones XOR (“+”) y AND (“*”).
\begin{quote}\begin{description}
\sphinxlineitem{Parameters}\begin{itemize}
\item {} 
\sphinxAtStartPar
\sphinxstyleliteralstrong{\sphinxupquote{coeff\_matrix\_in}} (\sphinxstyleliteralemphasis{\sphinxupquote{list}}\sphinxstyleliteralemphasis{\sphinxupquote{ of }}\sphinxstyleliteralemphasis{\sphinxupquote{list}}\sphinxstyleliteralemphasis{\sphinxupquote{ of }}\sphinxstyleliteralemphasis{\sphinxupquote{bool}}) \textendash{} Matriz booleana que representa los coeficientes del sistema de ecuaciones.

\item {} 
\sphinxAtStartPar
\sphinxstyleliteralstrong{\sphinxupquote{const\_vector\_in}} (\sphinxstyleliteralemphasis{\sphinxupquote{list}}\sphinxstyleliteralemphasis{\sphinxupquote{ of }}\sphinxstyleliteralemphasis{\sphinxupquote{bool}}) \textendash{} Vector booleano que representa el término independiente de cada ecuación.

\end{itemize}

\sphinxlineitem{Returns}
\sphinxAtStartPar
Matriz booleana triangular.

\sphinxlineitem{Return type}
\sphinxAtStartPar
numpy.ndarray

\end{description}\end{quote}

\end{fulllineitems}

\index{hamming() (in module myutils.bool)@\spxentry{hamming()}\spxextra{in module myutils.bool}}

\begin{fulllineitems}
\phantomsection\label{\detokenize{myutils:myutils.bool.hamming}}
\pysigstartsignatures
\pysiglinewithargsret{\sphinxcode{\sphinxupquote{myutils.bool.}}\sphinxbfcode{\sphinxupquote{hamming}}}{\sphinxparam{\DUrole{n}{in1}}\sphinxparamcomma \sphinxparam{\DUrole{n}{in2}}}{}
\pysigstopsignatures
\sphinxAtStartPar
Calcula la distancia de Hamming (medida en bits) entre dos entradas.
\begin{quote}\begin{description}
\sphinxlineitem{Parameters}\begin{itemize}
\item {} 
\sphinxAtStartPar
\sphinxstyleliteralstrong{\sphinxupquote{in1}} (\sphinxstyleliteralemphasis{\sphinxupquote{list}}\sphinxstyleliteralemphasis{\sphinxupquote{ of }}\sphinxstyleliteralemphasis{\sphinxupquote{bool}}) \textendash{} Primer vector booleano de entrada.

\item {} 
\sphinxAtStartPar
\sphinxstyleliteralstrong{\sphinxupquote{in2}} (\sphinxstyleliteralemphasis{\sphinxupquote{list}}\sphinxstyleliteralemphasis{\sphinxupquote{ of }}\sphinxstyleliteralemphasis{\sphinxupquote{bool}}) \textendash{} Segundo vector booleano de entrada.

\end{itemize}

\sphinxlineitem{Returns}
\sphinxAtStartPar
Distancia de Hamming entre las dos entradas (medida en bits).

\sphinxlineitem{Return type}
\sphinxAtStartPar
int

\end{description}\end{quote}

\end{fulllineitems}

\index{invert\_matrix\_gf2() (in module myutils.bool)@\spxentry{invert\_matrix\_gf2()}\spxextra{in module myutils.bool}}

\begin{fulllineitems}
\phantomsection\label{\detokenize{myutils:myutils.bool.invert_matrix_gf2}}
\pysigstartsignatures
\pysiglinewithargsret{\sphinxcode{\sphinxupquote{myutils.bool.}}\sphinxbfcode{\sphinxupquote{invert\_matrix\_gf2}}}{\sphinxparam{\DUrole{n}{matrix\_in}}}{}
\pysigstopsignatures
\sphinxAtStartPar
Devuelve la inversa de una matriz booleana en un cuerpo finito con las operaciones XOR (“+”) y AND (“*”).
\begin{quote}\begin{description}
\sphinxlineitem{Parameters}
\sphinxAtStartPar
\sphinxstyleliteralstrong{\sphinxupquote{matrix\_in}} (\sphinxstyleliteralemphasis{\sphinxupquote{list}}\sphinxstyleliteralemphasis{\sphinxupquote{ of }}\sphinxstyleliteralemphasis{\sphinxupquote{list}}\sphinxstyleliteralemphasis{\sphinxupquote{ of }}\sphinxstyleliteralemphasis{\sphinxupquote{bool}}) \textendash{} Matriz booleana a invertir.

\sphinxlineitem{Returns}
\sphinxAtStartPar
Matriz booleana inversa.

\sphinxlineitem{Return type}
\sphinxAtStartPar
numpy.ndarray

\end{description}\end{quote}

\end{fulllineitems}

\index{matrix\_multiply\_gf2() (in module myutils.bool)@\spxentry{matrix\_multiply\_gf2()}\spxextra{in module myutils.bool}}

\begin{fulllineitems}
\phantomsection\label{\detokenize{myutils:myutils.bool.matrix_multiply_gf2}}
\pysigstartsignatures
\pysiglinewithargsret{\sphinxcode{\sphinxupquote{myutils.bool.}}\sphinxbfcode{\sphinxupquote{matrix\_multiply\_gf2}}}{\sphinxparam{\DUrole{n}{matrix1\_in}}\sphinxparamcomma \sphinxparam{\DUrole{n}{matrix2\_in}}}{}
\pysigstopsignatures
\sphinxAtStartPar
Multiplica un par de matrices booleanas en un cuerpo finito con las operaciones XOR (“+”) y AND (“*”).

\sphinxAtStartPar
Esta función puede utilizarse también para multiplicar vectores por matrices, gestionando automáticamente
la disposición del vector: fila si va delante de la matriz, columna si va detrás. Así, si ‘matrix1\_in’ y
‘matrix2\_in’ son dos vectores, el resultado será el producto vectorial.
\begin{quote}\begin{description}
\sphinxlineitem{Parameters}\begin{itemize}
\item {} 
\sphinxAtStartPar
\sphinxstyleliteralstrong{\sphinxupquote{matrix1\_in}} (\sphinxstyleliteralemphasis{\sphinxupquote{list}}\sphinxstyleliteralemphasis{\sphinxupquote{ of }}\sphinxstyleliteralemphasis{\sphinxupquote{list}}\sphinxstyleliteralemphasis{\sphinxupquote{ of }}\sphinxstyleliteralemphasis{\sphinxupquote{bool}}) \textendash{} Primera matriz booleana.

\item {} 
\sphinxAtStartPar
\sphinxstyleliteralstrong{\sphinxupquote{matrix2\_in}} (\sphinxstyleliteralemphasis{\sphinxupquote{list}}\sphinxstyleliteralemphasis{\sphinxupquote{ of }}\sphinxstyleliteralemphasis{\sphinxupquote{list}}\sphinxstyleliteralemphasis{\sphinxupquote{ of }}\sphinxstyleliteralemphasis{\sphinxupquote{bool}}) \textendash{} Segunda matriz booleana.

\end{itemize}

\sphinxlineitem{Returns}
\sphinxAtStartPar
Producto de las matrices o el producto vectorial en gf(2).

\sphinxlineitem{Return type}
\sphinxAtStartPar
numpy.ndarray

\end{description}\end{quote}

\end{fulllineitems}



\section{myutils.stats module}
\label{\detokenize{myutils:module-myutils.stats}}\label{\detokenize{myutils:myutils-stats-module}}\index{module@\spxentry{module}!myutils.stats@\spxentry{myutils.stats}}\index{myutils.stats@\spxentry{myutils.stats}!module@\spxentry{module}}\index{Dks\_montecarlo\_discrete() (in module myutils.stats)@\spxentry{Dks\_montecarlo\_discrete()}\spxextra{in module myutils.stats}}

\begin{fulllineitems}
\phantomsection\label{\detokenize{myutils:myutils.stats.Dks_montecarlo_discrete}}
\pysigstartsignatures
\pysiglinewithargsret{\sphinxcode{\sphinxupquote{myutils.stats.}}\sphinxbfcode{\sphinxupquote{Dks\_montecarlo\_discrete}}}{\sphinxparam{\DUrole{n}{model}}\sphinxparamcomma \sphinxparam{\DUrole{n}{fit}}\sphinxparamcomma \sphinxparam{\DUrole{n}{N}}\sphinxparamcomma \sphinxparam{\DUrole{n}{verbose}\DUrole{o}{=}\DUrole{default_value}{True}}\sphinxparamcomma \sphinxparam{\DUrole{o}{**}\DUrole{n}{kwargs}}}{}
\pysigstopsignatures
\sphinxAtStartPar
Calcula la distribución del estadístico KS de un modelo discreto en comparación con una distribución teórica.
\begin{quote}\begin{description}
\sphinxlineitem{Parameters}\begin{itemize}
\item {} 
\sphinxAtStartPar
\sphinxstyleliteralstrong{\sphinxupquote{model}} (\sphinxstyleliteralemphasis{\sphinxupquote{callable}}) \textendash{} Función aleatoria a contrastar. El primer parámetro que debe aceptar es ‘N’, seguido de un número arbitrario de
parámetros que son pasados mediante ‘kwargs’. Devuelve un array de valores aleatorios.

\item {} 
\sphinxAtStartPar
\sphinxstyleliteralstrong{\sphinxupquote{fit}} (\sphinxstyleliteralemphasis{\sphinxupquote{list}}\sphinxstyleliteralemphasis{\sphinxupquote{ of }}\sphinxstyleliteralemphasis{\sphinxupquote{float}}) \textendash{} Lista de valores que representa la distribución de probabilidad contra la cual se calcula la distribución de KS.

\item {} 
\sphinxAtStartPar
\sphinxstyleliteralstrong{\sphinxupquote{N}} (\sphinxstyleliteralemphasis{\sphinxupquote{int}}) \textendash{} Número de veces que se repite el cálculo de Dks.

\item {} 
\sphinxAtStartPar
\sphinxstyleliteralstrong{\sphinxupquote{verbose}} (\sphinxstyleliteralemphasis{\sphinxupquote{bool}}\sphinxstyleliteralemphasis{\sphinxupquote{, }}\sphinxstyleliteralemphasis{\sphinxupquote{optional}}) \textendash{} Si es True, imprime el progreso. Por defecto es True.

\item {} 
\sphinxAtStartPar
\sphinxstyleliteralstrong{\sphinxupquote{kwargs}} (\sphinxstyleliteralemphasis{\sphinxupquote{dict}}) \textendash{} Parámetros a pasar a ‘model’.

\end{itemize}

\sphinxlineitem{Returns}
\sphinxAtStartPar
Lista de valores Dks.

\sphinxlineitem{Return type}
\sphinxAtStartPar
list of float

\end{description}\end{quote}
\subsubsection*{Notes}

\sphinxAtStartPar
Esta función calcula la distribución del estadístico KS de un modelo ‘model’ que produce ‘N’ valores aleatorios frente a
una curva teórica ‘fit’, representada como una lista de valores. Devuelve una lista de estos índices KS.

\sphinxAtStartPar
El método ‘model’ admite una cantidad arbitraria de parámetros pasados mediante ‘kwargs’.

\end{fulllineitems}

\index{bin\_rv\_cont() (in module myutils.stats)@\spxentry{bin\_rv\_cont()}\spxextra{in module myutils.stats}}

\begin{fulllineitems}
\phantomsection\label{\detokenize{myutils:myutils.stats.bin_rv_cont}}
\pysigstartsignatures
\pysiglinewithargsret{\sphinxcode{\sphinxupquote{myutils.stats.}}\sphinxbfcode{\sphinxupquote{bin\_rv\_cont}}}{\sphinxparam{\DUrole{n}{rv\_continuous}}\sphinxparamcomma \sphinxparam{\DUrole{n}{bins\_in}}\sphinxparamcomma \sphinxparam{\DUrole{o}{*}\DUrole{n}{args}}}{}
\pysigstopsignatures
\sphinxAtStartPar
Calcula la probabilidad acumulada de un objeto ‘rv\_continuous’ de scipy en los extremos dados por ‘bins’.
\begin{quote}\begin{description}
\sphinxlineitem{Parameters}\begin{itemize}
\item {} 
\sphinxAtStartPar
\sphinxstyleliteralstrong{\sphinxupquote{rv\_continuous}} (\sphinxstyleliteralemphasis{\sphinxupquote{scipy.stats.rv\_continuous}}) \textendash{} Objeto ‘rv\_continuous’ de scipy.

\item {} 
\sphinxAtStartPar
\sphinxstyleliteralstrong{\sphinxupquote{bins\_in}} (\sphinxstyleliteralemphasis{\sphinxupquote{array\_like}}) \textendash{} Bineado del eje de abscisas. Debe ser una lista de ‘N\_bins+1’ elementos, conteniendo los extremos de cada bin.

\item {} 
\sphinxAtStartPar
\sphinxstyleliteralstrong{\sphinxupquote{args}} (\sphinxstyleliteralemphasis{\sphinxupquote{list}}\sphinxstyleliteralemphasis{\sphinxupquote{, }}\sphinxstyleliteralemphasis{\sphinxupquote{optional}}) \textendash{} Parámetros extra para pasar a ‘rv\_continuous.cdf’, distintos del primer argumento (‘x’).

\end{itemize}

\sphinxlineitem{Returns}
\sphinxAtStartPar
Arreglo de tamaño ‘N\_bin’ elementos que contiene la probabilidad acumulada en los extremos dados por ‘bins’.
El arreglo de salida está normalizado respecto a la suma de sus elementos.

\sphinxlineitem{Return type}
\sphinxAtStartPar
numpy.ndarray

\end{description}\end{quote}
\subsubsection*{Notes}

\sphinxAtStartPar
Esta función toma un objeto ‘rv\_continuous’ de scipy y una lista ‘bins\_in’ que contiene el bineado del eje de abscisas.
Calcula la probabilidad acumulada en los extremos dados por ‘bins’ utilizando ‘rv\_continuous.cdf’ y devuelve un numpy array
normalizado con las probabilidades acumuladas.

\sphinxAtStartPar
El argumento ‘args’ se utiliza para pasar parámetros extra distintos al primero (‘x’) a ‘rv\_continuous.cdf’.

\end{fulllineitems}

\index{bin\_rv\_discrete() (in module myutils.stats)@\spxentry{bin\_rv\_discrete()}\spxextra{in module myutils.stats}}

\begin{fulllineitems}
\phantomsection\label{\detokenize{myutils:myutils.stats.bin_rv_discrete}}
\pysigstartsignatures
\pysiglinewithargsret{\sphinxcode{\sphinxupquote{myutils.stats.}}\sphinxbfcode{\sphinxupquote{bin\_rv\_discrete}}}{\sphinxparam{\DUrole{n}{rv\_discrete}}\sphinxparamcomma \sphinxparam{\DUrole{n}{bins\_in}}\sphinxparamcomma \sphinxparam{\DUrole{o}{*}\DUrole{n}{args}}}{}
\pysigstopsignatures
\sphinxAtStartPar
Calcula la probabilidad acumulada de un objeto ‘rv\_discrete’ de scipy en los extremos dados por ‘bins’.
\begin{quote}\begin{description}
\sphinxlineitem{Parameters}\begin{itemize}
\item {} 
\sphinxAtStartPar
\sphinxstyleliteralstrong{\sphinxupquote{rv\_discrete}} (\sphinxstyleliteralemphasis{\sphinxupquote{scipy.stats.rv\_discrete}}) \textendash{} Objeto ‘rv\_discrete’ de scipy.

\item {} 
\sphinxAtStartPar
\sphinxstyleliteralstrong{\sphinxupquote{bins\_in}} (\sphinxstyleliteralemphasis{\sphinxupquote{array\_like}}) \textendash{} Bineado del eje de abscisas. Debe ser una lista de ‘N\_bins+1’ elementos, conteniendo los extremos de cada bin.

\item {} 
\sphinxAtStartPar
\sphinxstyleliteralstrong{\sphinxupquote{args}} (\sphinxstyleliteralemphasis{\sphinxupquote{list}}\sphinxstyleliteralemphasis{\sphinxupquote{, }}\sphinxstyleliteralemphasis{\sphinxupquote{optional}}) \textendash{} Parámetros extra para pasar a ‘rv\_discrete’, distintos del primer argumento (‘k’).

\end{itemize}

\sphinxlineitem{Returns}
\sphinxAtStartPar
Arreglo de tamaño ‘N\_bin’ elementos que contiene la probabilidad acumulada en los extremos dados por ‘bins’.
El arreglo de salida está normalizado respecto a la suma de sus elementos.

\sphinxlineitem{Return type}
\sphinxAtStartPar
numpy.ndarray

\end{description}\end{quote}
\subsubsection*{Notes}

\sphinxAtStartPar
Esta función toma un objeto ‘rv\_discrete’ de scipy y una lista ‘bins\_in’ que contiene el bineado del eje de abscisas.
Calcula la probabilidad acumulada en los extremos dados por ‘bins’ y devuelve un numpy array normalizado con las probabilidades acumuladas.

\sphinxAtStartPar
El argumento ‘args’ se utiliza para pasar parámetros extra distintos del primero (‘k’) a ‘rv\_discrete’.

\end{fulllineitems}

\index{chisq\_gof\_discrete() (in module myutils.stats)@\spxentry{chisq\_gof\_discrete()}\spxextra{in module myutils.stats}}

\begin{fulllineitems}
\phantomsection\label{\detokenize{myutils:myutils.stats.chisq_gof_discrete}}
\pysigstartsignatures
\pysiglinewithargsret{\sphinxcode{\sphinxupquote{myutils.stats.}}\sphinxbfcode{\sphinxupquote{chisq\_gof\_discrete}}}{\sphinxparam{\DUrole{n}{obs\_data}}\sphinxparamcomma \sphinxparam{\DUrole{n}{model}}\sphinxparamcomma \sphinxparam{\DUrole{n}{alpha}\DUrole{o}{=}\DUrole{default_value}{0.05}}\sphinxparamcomma \sphinxparam{\DUrole{n}{plothist}\DUrole{o}{=}\DUrole{default_value}{False}}\sphinxparamcomma \sphinxparam{\DUrole{o}{**}\DUrole{n}{kwargs}}}{}
\pysigstopsignatures
\sphinxAtStartPar
Realiza un test chi\textasciicircum{}2 de bondad del ajuste entre los datos observados y un modelo
estocástico discreto (una función pmf \sphinxhyphen{}probability mass function\sphinxhyphen{}) que hipotéticamente
genera dichos datos.
\begin{quote}\begin{description}
\sphinxlineitem{Parameters}\begin{itemize}
\item {} 
\sphinxAtStartPar
\sphinxstyleliteralstrong{\sphinxupquote{obs\_data}} (\sphinxstyleliteralemphasis{\sphinxupquote{array\_like}}) \textendash{} Vector de datos observados.

\item {} 
\sphinxAtStartPar
\sphinxstyleliteralstrong{\sphinxupquote{model}} (\sphinxstyleliteralemphasis{\sphinxupquote{callable}}) \textendash{} Función pmf (probability mass function) que representa el modelo estocástico discreto.
El primer argumento siempre debe ser la variable independiente (aleatoria).

\item {} 
\sphinxAtStartPar
\sphinxstyleliteralstrong{\sphinxupquote{alpha}} (\sphinxstyleliteralemphasis{\sphinxupquote{float}}\sphinxstyleliteralemphasis{\sphinxupquote{, }}\sphinxstyleliteralemphasis{\sphinxupquote{optional}}) \textendash{} Nivel de significancia para el test chi\textasciicircum{}2. Por defecto es 0.05.

\item {} 
\sphinxAtStartPar
\sphinxstyleliteralstrong{\sphinxupquote{plothist}} (\sphinxstyleliteralemphasis{\sphinxupquote{bool}}\sphinxstyleliteralemphasis{\sphinxupquote{, }}\sphinxstyleliteralemphasis{\sphinxupquote{optional}}) \textendash{} Indica si se debe graficar el histograma de los datos. Por defecto es False.

\item {} 
\sphinxAtStartPar
\sphinxstyleliteralstrong{\sphinxupquote{kwargs}} (\sphinxstyleliteralemphasis{\sphinxupquote{dict}}) \textendash{} Opciones adicionales para pasar parámetros a la función pmf introducida como modelo.

\end{itemize}

\sphinxlineitem{Returns}
\sphinxAtStartPar
Un diccionario con los campos:
\sphinxhyphen{} ‘chisq’: Valor del estadístico chi\textasciicircum{}2.
\sphinxhyphen{} ‘pvalue’: Valor tal que la probabilidad de obtener ‘chisq’ es al menos pvalue.
\sphinxhyphen{} ‘dof’: Número de grados de libertad de la distribución chi\textasciicircum{}2.
\sphinxhyphen{} ‘testpass’: Booleano que indica si el test ha sido superado en función de ‘alpha’.

\sphinxlineitem{Return type}
\sphinxAtStartPar
dict

\end{description}\end{quote}
\subsubsection*{Notes}

\sphinxAtStartPar
Esta función calcula un test chi\textasciicircum{}2 de bondad del ajuste entre los datos observados y un modelo
estocástico discreto proporcionado como una función pmf. Se genera un histograma de los datos
y se calculan los valores esperados de probabilidad para cada bin. Luego se aplica la función
‘chisquare’ de scipy para realizar el test chi\textasciicircum{}2.

\sphinxAtStartPar
El parámetro ‘kwargs’ se puede utilizar para pasar parámetros adicionales a la función pmf del modelo.

\end{fulllineitems}

\index{fit\_rv\_cont() (in module myutils.stats)@\spxentry{fit\_rv\_cont()}\spxextra{in module myutils.stats}}

\begin{fulllineitems}
\phantomsection\label{\detokenize{myutils:myutils.stats.fit_rv_cont}}
\pysigstartsignatures
\pysiglinewithargsret{\sphinxcode{\sphinxupquote{myutils.stats.}}\sphinxbfcode{\sphinxupquote{fit\_rv\_cont}}}{\sphinxparam{\DUrole{n}{data}}\sphinxparamcomma \sphinxparam{\DUrole{n}{rv\_cont}}\sphinxparamcomma \sphinxparam{\DUrole{n}{bins}\DUrole{o}{=}\DUrole{default_value}{10}}\sphinxparamcomma \sphinxparam{\DUrole{n}{alpha}\DUrole{o}{=}\DUrole{default_value}{0.05}}\sphinxparamcomma \sphinxparam{\DUrole{n}{plot}\DUrole{o}{=}\DUrole{default_value}{False}}}{}
\pysigstopsignatures
\sphinxAtStartPar
Encuentra la distribución ‘rv\_continuous’ que mejor ajusta un conjunto de datos ‘data’ y calcula el ‘valor p’ correspondiente a una significancia ‘alpha’.
\begin{quote}\begin{description}
\sphinxlineitem{Parameters}\begin{itemize}
\item {} 
\sphinxAtStartPar
\sphinxstyleliteralstrong{\sphinxupquote{data}} (\sphinxstyleliteralemphasis{\sphinxupquote{list}}\sphinxstyleliteralemphasis{\sphinxupquote{ of }}\sphinxstyleliteralemphasis{\sphinxupquote{float}}) \textendash{} Vector con los datos a ajustar.

\item {} 
\sphinxAtStartPar
\sphinxstyleliteralstrong{\sphinxupquote{rv\_cont}} (\sphinxstyleliteralemphasis{\sphinxupquote{scipy.stats.rv\_continuous}}) \textendash{} Objeto ‘scipy.stats.rv\_continuous’ que define una distribución de probabilidad continua que se ajustará a los datos ‘data’.

\item {} 
\sphinxAtStartPar
\sphinxstyleliteralstrong{\sphinxupquote{bins}} (\sphinxstyleliteralemphasis{\sphinxupquote{int}}\sphinxstyleliteralemphasis{\sphinxupquote{, }}\sphinxstyleliteralemphasis{\sphinxupquote{optional}}) \textendash{} Número de cajas para el histograma de ‘data’. Por defecto es 10.

\item {} 
\sphinxAtStartPar
\sphinxstyleliteralstrong{\sphinxupquote{alpha}} (\sphinxstyleliteralemphasis{\sphinxupquote{float}}\sphinxstyleliteralemphasis{\sphinxupquote{, }}\sphinxstyleliteralemphasis{\sphinxupquote{optional}}) \textendash{} Valor de significancia ‘alpha’ para el cual se calcula el valor p. Por defecto es 0.05.

\item {} 
\sphinxAtStartPar
\sphinxstyleliteralstrong{\sphinxupquote{plot}} (\sphinxstyleliteralemphasis{\sphinxupquote{bool}}\sphinxstyleliteralemphasis{\sphinxupquote{, }}\sphinxstyleliteralemphasis{\sphinxupquote{optional}}) \textendash{} Si es True, pinta el histograma y superpone la curva encontrada. La función no ejecuta ‘show()’, de forma que el usuario puede recabar la figura externamente
con ‘gca()’ y ‘gcf()’, y editarla antes de representarla. Por defecto es False.

\end{itemize}

\sphinxlineitem{Returns}
\sphinxAtStartPar
Una lista que contiene tres elementos:
\sphinxhyphen{} El primero es una lista de los parámetros que mejor ajustan la distribución (en el mismo orden en que se definen en la función ‘rv\_cont.pdf’).
\sphinxhyphen{} El segundo es un nparray con el histograma.
\sphinxhyphen{} El tercero es un nparray con el bineado de los datos de entrada.

\sphinxlineitem{Return type}
\sphinxAtStartPar
list

\end{description}\end{quote}
\subsubsection*{Notes}

\sphinxAtStartPar
Esta función es un wrapper para encontrar la distribución ‘rv\_continuous’ que mejor ajusta un conjunto de valores ‘data’ utilizando la máxima verosimilitud.
Calcula el ‘valor p’ correspondiente a una significancia ‘alpha’. Resulta útil para distribuciones obtenidas por simulación, que carecen de una función
densidad teórica.

\end{fulllineitems}

\index{get\_area\_fraction() (in module myutils.stats)@\spxentry{get\_area\_fraction()}\spxextra{in module myutils.stats}}

\begin{fulllineitems}
\phantomsection\label{\detokenize{myutils:myutils.stats.get_area_fraction}}
\pysigstartsignatures
\pysiglinewithargsret{\sphinxcode{\sphinxupquote{myutils.stats.}}\sphinxbfcode{\sphinxupquote{get\_area\_fraction}}}{\sphinxparam{\DUrole{n}{hist}}\sphinxparamcomma \sphinxparam{\DUrole{n}{p}\DUrole{o}{=}\DUrole{default_value}{0.9}}}{}
\pysigstopsignatures
\sphinxAtStartPar
Encuentra los extremos ‘xmin’ y ‘xmax’ de una lista ‘hist’ de números (no necesariamente normalizada)
de tal manera que el intervalo ‘hist{[}xmin:xmax{]}’ contenga al menos una fracción ‘p’ de la suma acumulada
de los elementos de la lista.
\begin{quote}\begin{description}
\sphinxlineitem{Parameters}\begin{itemize}
\item {} 
\sphinxAtStartPar
\sphinxstyleliteralstrong{\sphinxupquote{hist}} (\sphinxstyleliteralemphasis{\sphinxupquote{list}}\sphinxstyleliteralemphasis{\sphinxupquote{ or }}\sphinxstyleliteralemphasis{\sphinxupquote{array\_like}}) \textendash{} Lista de números.

\item {} 
\sphinxAtStartPar
\sphinxstyleliteralstrong{\sphinxupquote{p}} (\sphinxstyleliteralemphasis{\sphinxupquote{float}}\sphinxstyleliteralemphasis{\sphinxupquote{, }}\sphinxstyleliteralemphasis{\sphinxupquote{optional}}) \textendash{} Fracción mínima deseada. Por defecto es 0.9.

\end{itemize}

\sphinxlineitem{Returns}
\sphinxAtStartPar
\begin{itemize}
\item {} 
\sphinxAtStartPar
\sphinxstylestrong{xmin} (\sphinxstyleemphasis{int}) \textendash{} Índice del extremo izquierdo del intervalo.

\item {} 
\sphinxAtStartPar
\sphinxstylestrong{xmax} (\sphinxstyleemphasis{int}) \textendash{} Índice del extremo derecho del intervalo.

\end{itemize}


\end{description}\end{quote}
\subsubsection*{Notes}

\sphinxAtStartPar
Esta función busca los extremos ‘xmin’ y ‘xmax’ en la lista ‘hist’ de números, de tal manera que el intervalo
‘hist{[}xmin:xmax{]}’ contenga al menos una fracción ‘p’ de la suma acumulada de los elementos de la lista. La búsqueda
comienza desde el índice correspondiente al valor máximo de la lista y se desplaza simétricamente hacia ambos extremos.

\end{fulllineitems}

\index{get\_hist\_smooth() (in module myutils.stats)@\spxentry{get\_hist\_smooth()}\spxextra{in module myutils.stats}}

\begin{fulllineitems}
\phantomsection\label{\detokenize{myutils:myutils.stats.get_hist_smooth}}
\pysigstartsignatures
\pysiglinewithargsret{\sphinxcode{\sphinxupquote{myutils.stats.}}\sphinxbfcode{\sphinxupquote{get\_hist\_smooth}}}{\sphinxparam{\DUrole{n}{data}}\sphinxparamcomma \sphinxparam{\DUrole{n}{p}\DUrole{o}{=}\DUrole{default_value}{0.9}}\sphinxparamcomma \sphinxparam{\DUrole{n}{x0}\DUrole{o}{=}\DUrole{default_value}{nan}}\sphinxparamcomma \sphinxparam{\DUrole{n}{x1}\DUrole{o}{=}\DUrole{default_value}{nan}}}{}
\pysigstopsignatures
\sphinxAtStartPar
Genera un histograma suavizado a partir de un conjunto de datos ‘data’.
\begin{quote}\begin{description}
\sphinxlineitem{Parameters}\begin{itemize}
\item {} 
\sphinxAtStartPar
\sphinxstyleliteralstrong{\sphinxupquote{data}} (\sphinxstyleliteralemphasis{\sphinxupquote{array\_like}}) \textendash{} Conjunto de datos para representar en forma de histograma.

\item {} 
\sphinxAtStartPar
\sphinxstyleliteralstrong{\sphinxupquote{p}} (\sphinxstyleliteralemphasis{\sphinxupquote{float}}\sphinxstyleliteralemphasis{\sphinxupquote{, }}\sphinxstyleliteralemphasis{\sphinxupquote{optional}}) \textendash{} Fracción (tanto por uno) de área bajo la cual se busca que el histograma no presente máximos relativos.
Por defecto es 0.9.

\item {} 
\sphinxAtStartPar
\sphinxstyleliteralstrong{\sphinxupquote{x0}} (\sphinxstyleliteralemphasis{\sphinxupquote{float}}\sphinxstyleliteralemphasis{\sphinxupquote{, }}\sphinxstyleliteralemphasis{\sphinxupquote{optional}}) \textendash{} Extremo inferior del rango del histograma. Por defecto es NaN (se calcula automáticamente a partir de ‘data’).

\item {} 
\sphinxAtStartPar
\sphinxstyleliteralstrong{\sphinxupquote{x1}} (\sphinxstyleliteralemphasis{\sphinxupquote{float}}\sphinxstyleliteralemphasis{\sphinxupquote{, }}\sphinxstyleliteralemphasis{\sphinxupquote{optional}}) \textendash{} Extremo superior del rango del histograma. Por defecto es NaN (se calcula automáticamente a partir de ‘data’).

\end{itemize}

\sphinxlineitem{Returns}
\sphinxAtStartPar
Histograma suavizado generado.

\sphinxlineitem{Return type}
\sphinxAtStartPar
numpy.ndarray

\end{description}\end{quote}
\subsubsection*{Notes}

\sphinxAtStartPar
Esta función devuelve un histograma con el número de bins adecuado para que no presente máximos locales,
lo cual resulta en un aspecto más suave. El parámetro ‘p’ representa la fracción de área bajo la cual
se busca que el histograma no presente máximos relativos, medido simétricamente desde el punto máximo
del histograma.

\sphinxAtStartPar
Los parámetros ‘x0’ y ‘x1’ son opcionales y se utilizan para definir el rango del histograma; si no se
proporcionan, se calculan automáticamente a partir de los valores mínimo y máximo de ‘data’.

\end{fulllineitems}

\index{intervalo\_int() (in module myutils.stats)@\spxentry{intervalo\_int()}\spxextra{in module myutils.stats}}

\begin{fulllineitems}
\phantomsection\label{\detokenize{myutils:myutils.stats.intervalo_int}}
\pysigstartsignatures
\pysiglinewithargsret{\sphinxcode{\sphinxupquote{myutils.stats.}}\sphinxbfcode{\sphinxupquote{intervalo\_int}}}{\sphinxparam{\DUrole{n}{inf}}\sphinxparamcomma \sphinxparam{\DUrole{n}{sup}}}{}
\pysigstopsignatures
\sphinxAtStartPar
Esta función auxiliar devuelve los extremos de números enteros dado un intervalo real.
\begin{quote}\begin{description}
\sphinxlineitem{Parameters}\begin{itemize}
\item {} 
\sphinxAtStartPar
\sphinxstyleliteralstrong{\sphinxupquote{inf}} (\sphinxstyleliteralemphasis{\sphinxupquote{float}}) \textendash{} Extremo inferior del intervalo.

\item {} 
\sphinxAtStartPar
\sphinxstyleliteralstrong{\sphinxupquote{sup}} (\sphinxstyleliteralemphasis{\sphinxupquote{float}}) \textendash{} Extremo superior del intervalo.

\end{itemize}

\sphinxlineitem{Returns}
\sphinxAtStartPar
Rango de enteros entre los extremos \sphinxtitleref{inf} y \sphinxtitleref{sup}.

\sphinxlineitem{Return type}
\sphinxAtStartPar
list

\end{description}\end{quote}

\end{fulllineitems}



\section{myutils.tensor module}
\label{\detokenize{myutils:module-myutils.tensor}}\label{\detokenize{myutils:myutils-tensor-module}}\index{module@\spxentry{module}!myutils.tensor@\spxentry{myutils.tensor}}\index{myutils.tensor@\spxentry{myutils.tensor}!module@\spxentry{module}}
\sphinxAtStartPar
Este módulo permite definir nombres para los ejes
de un numpyarray.
\index{Tensor (class in myutils.tensor)@\spxentry{Tensor}\spxextra{class in myutils.tensor}}

\begin{fulllineitems}
\phantomsection\label{\detokenize{myutils:myutils.tensor.Tensor}}
\pysigstartsignatures
\pysiglinewithargsret{\sphinxbfcode{\sphinxupquote{class\DUrole{w}{ }}}\sphinxcode{\sphinxupquote{myutils.tensor.}}\sphinxbfcode{\sphinxupquote{Tensor}}}{\sphinxparam{\DUrole{n}{array}}\sphinxparamcomma \sphinxparam{\DUrole{n}{axis}\DUrole{o}{=}\DUrole{default_value}{False}}}{}
\pysigstopsignatures
\sphinxAtStartPar
Bases: \sphinxcode{\sphinxupquote{object}}

\sphinxAtStartPar
Un objeto que es esencialmente un nparray junto con una variable ‘self.axis’ que permite indexar los ejes del array mediante un nombre natural, en lugar de un índice entero.
\begin{quote}\begin{description}
\sphinxlineitem{Parameters}\begin{itemize}
\item {} 
\sphinxAtStartPar
\sphinxstyleliteralstrong{\sphinxupquote{array}} (\sphinxstyleliteralemphasis{\sphinxupquote{numpy.ndarray}}) \textendash{} Arreglo que contiene los datos y la forma del tensor.

\item {} 
\sphinxAtStartPar
\sphinxstyleliteralstrong{\sphinxupquote{axis}} (\sphinxstyleliteralemphasis{\sphinxupquote{list}}\sphinxstyleliteralemphasis{\sphinxupquote{ of }}\sphinxstyleliteralemphasis{\sphinxupquote{strings}}\sphinxstyleliteralemphasis{\sphinxupquote{, }}\sphinxstyleliteralemphasis{\sphinxupquote{optional}}) \textendash{} Nombres de los ejes. El orden en que se proporcionan los nombres se correlaciona
con el correspondiente eje del parámetro ‘array’. Si no se introduce una lista de nombres en esta variable,
se asignarán por defecto como nombres el índice de cada eje (empezando en ‘0’).

\end{itemize}

\end{description}\end{quote}
\index{array (myutils.tensor.Tensor attribute)@\spxentry{array}\spxextra{myutils.tensor.Tensor attribute}}

\begin{fulllineitems}
\phantomsection\label{\detokenize{myutils:myutils.tensor.Tensor.array}}
\pysigstartsignatures
\pysigline{\sphinxbfcode{\sphinxupquote{array}}}
\pysigstopsignatures
\sphinxAtStartPar
Arreglo que contiene los datos y la forma del tensor.
\begin{quote}\begin{description}
\sphinxlineitem{Type}
\sphinxAtStartPar
numpy.ndarray

\end{description}\end{quote}

\end{fulllineitems}

\index{shape (myutils.tensor.Tensor attribute)@\spxentry{shape}\spxextra{myutils.tensor.Tensor attribute}}

\begin{fulllineitems}
\phantomsection\label{\detokenize{myutils:myutils.tensor.Tensor.shape}}
\pysigstartsignatures
\pysigline{\sphinxbfcode{\sphinxupquote{shape}}}
\pysigstopsignatures
\sphinxAtStartPar
Forma del tensor (igual que en un nparray).
\begin{quote}\begin{description}
\sphinxlineitem{Type}
\sphinxAtStartPar
tuple

\end{description}\end{quote}

\end{fulllineitems}

\index{axis (myutils.tensor.Tensor attribute)@\spxentry{axis}\spxextra{myutils.tensor.Tensor attribute}}

\begin{fulllineitems}
\phantomsection\label{\detokenize{myutils:myutils.tensor.Tensor.axis}}
\pysigstartsignatures
\pysigline{\sphinxbfcode{\sphinxupquote{axis}}}
\pysigstopsignatures
\sphinxAtStartPar
Nombres de los ejes. El i\sphinxhyphen{}ésimo elemento contiene el nombre del i\sphinxhyphen{}ésimo eje.
Si no se proporciona, se asignan nombres predeterminados en forma de índices.
\begin{quote}\begin{description}
\sphinxlineitem{Type}
\sphinxAtStartPar
list of strings

\end{description}\end{quote}

\end{fulllineitems}

\index{item() (myutils.tensor.Tensor method)@\spxentry{item()}\spxextra{myutils.tensor.Tensor method}}

\begin{fulllineitems}
\phantomsection\label{\detokenize{myutils:myutils.tensor.Tensor.item}}
\pysigstartsignatures
\pysiglinewithargsret{\sphinxbfcode{\sphinxupquote{item}}}{\sphinxparam{\DUrole{o}{**}\DUrole{n}{axes}}}{}
\pysigstopsignatures
\sphinxAtStartPar
Esta función toma por argumento el nombre de cada eje igualado a un
índice, y devuelve el valor guardado en esa coordenada.
\begin{quote}\begin{description}
\sphinxlineitem{Parameters}
\sphinxAtStartPar
\sphinxstyleliteralstrong{\sphinxupquote{axes}} (\sphinxstyleliteralemphasis{\sphinxupquote{dict}}) \textendash{} Argumentos clave\sphinxhyphen{}valor donde las claves son los nombres de los ejes y
los valores son los índices para esos ejes.

\sphinxlineitem{Returns}
\sphinxAtStartPar
Valor almacenado en las coordenadas especificadas.

\sphinxlineitem{Return type}
\sphinxAtStartPar
object

\end{description}\end{quote}

\end{fulllineitems}

\index{mean() (myutils.tensor.Tensor method)@\spxentry{mean()}\spxextra{myutils.tensor.Tensor method}}

\begin{fulllineitems}
\phantomsection\label{\detokenize{myutils:myutils.tensor.Tensor.mean}}
\pysigstartsignatures
\pysiglinewithargsret{\sphinxbfcode{\sphinxupquote{mean}}}{\sphinxparam{\DUrole{n}{axe}}}{}
\pysigstopsignatures
\sphinxAtStartPar
Devuelve un Tensor comprimido en el eje ‘axe’, suprimiéndolo y calculando el promedio a lo largo de dicho eje.
\begin{quote}\begin{description}
\sphinxlineitem{Parameters}
\sphinxAtStartPar
\sphinxstyleliteralstrong{\sphinxupquote{axe}} (\sphinxstyleliteralemphasis{\sphinxupquote{int}}) \textendash{} Índice del eje a comprimir y calcular el promedio.

\sphinxlineitem{Returns}
\sphinxAtStartPar
Tensor con el eje ‘axe’ suprimido y cada punto contiene el promedio a lo largo de ese eje.

\sphinxlineitem{Return type}
\sphinxAtStartPar
{\hyperref[\detokenize{myutils:myutils.tensor.Tensor}]{\sphinxcrossref{Tensor}}}

\end{description}\end{quote}

\end{fulllineitems}

\index{save() (myutils.tensor.Tensor method)@\spxentry{save()}\spxextra{myutils.tensor.Tensor method}}

\begin{fulllineitems}
\phantomsection\label{\detokenize{myutils:myutils.tensor.Tensor.save}}
\pysigstartsignatures
\pysiglinewithargsret{\sphinxbfcode{\sphinxupquote{save}}}{\sphinxparam{\DUrole{n}{file\_name}}\sphinxparamcomma \sphinxparam{\DUrole{n}{fmt}\DUrole{o}{=}\DUrole{default_value}{\textquotesingle{}\%.18e\textquotesingle{}}}}{}
\pysigstopsignatures
\sphinxAtStartPar
Guarda el Tensor en un archivo de texto.
\begin{quote}\begin{description}
\sphinxlineitem{Parameters}\begin{itemize}
\item {} 
\sphinxAtStartPar
\sphinxstyleliteralstrong{\sphinxupquote{file\_name}} (\sphinxstyleliteralemphasis{\sphinxupquote{str}}) \textendash{} Nombre del archivo donde se guardará el Tensor.

\item {} 
\sphinxAtStartPar
\sphinxstyleliteralstrong{\sphinxupquote{fmt}} (\sphinxstyleliteralemphasis{\sphinxupquote{str}}\sphinxstyleliteralemphasis{\sphinxupquote{, }}\sphinxstyleliteralemphasis{\sphinxupquote{optional}}) \textendash{} Formato numérico de los datos. Sigue el mismo formato que el parámetro ‘fmt’
de numpy.savetxt.

\end{itemize}

\sphinxlineitem{Return type}
\sphinxAtStartPar
None

\end{description}\end{quote}

\end{fulllineitems}

\index{size() (myutils.tensor.Tensor method)@\spxentry{size()}\spxextra{myutils.tensor.Tensor method}}

\begin{fulllineitems}
\phantomsection\label{\detokenize{myutils:myutils.tensor.Tensor.size}}
\pysigstartsignatures
\pysiglinewithargsret{\sphinxbfcode{\sphinxupquote{size}}}{\sphinxparam{\DUrole{n}{axis}}}{}
\pysigstopsignatures
\sphinxAtStartPar
Esta función toma el nombre de un eje y devuelve su dimensión.
\begin{quote}\begin{description}
\sphinxlineitem{Parameters}
\sphinxAtStartPar
\sphinxstyleliteralstrong{\sphinxupquote{axis}} (\sphinxstyleliteralemphasis{\sphinxupquote{str}}) \textendash{} Nombre del eje del que se desea obtener la dimensión.

\sphinxlineitem{Returns}
\sphinxAtStartPar
Dimensión del eje especificado.

\sphinxlineitem{Return type}
\sphinxAtStartPar
int

\end{description}\end{quote}

\end{fulllineitems}

\index{slice() (myutils.tensor.Tensor method)@\spxentry{slice()}\spxextra{myutils.tensor.Tensor method}}

\begin{fulllineitems}
\phantomsection\label{\detokenize{myutils:myutils.tensor.Tensor.slice}}
\pysigstartsignatures
\pysiglinewithargsret{\sphinxbfcode{\sphinxupquote{slice}}}{\sphinxparam{\DUrole{o}{**}\DUrole{n}{axes}}}{}
\pysigstopsignatures
\sphinxAtStartPar
Toma una serie de ejes (por nombre), igualados a una porción ‘slice’. Esta porción puede ser
un entero, una lista de dos enteros, o una lista de tres enteros, y devuelve un Tensor reducido al ‘slice’
indicado (usando la misma notación que un slice estándar en Python: {[}x{]} es la x\sphinxhyphen{}ésima entrada, {[}x,y{]} son
todas entradas en {[}x,y), y {[}x,y,z{]} son todas las entradas en {[}x,y) tomadas en saltos de z). Esta función
es en realidad un wrapper a ‘take’, usando la nomenclatura de Python para hacer slicing.
\begin{quote}\begin{description}
\sphinxlineitem{Parameters}
\sphinxAtStartPar
\sphinxstyleliteralstrong{\sphinxupquote{axes}} (\sphinxstyleliteralemphasis{\sphinxupquote{dict}}) \textendash{} Argumentos clave\sphinxhyphen{}valor donde las claves son los nombres de los ejes y los valores
son objetos ‘slice’ o listas de enteros para realizar la porción en esos ejes.

\sphinxlineitem{Returns}
\sphinxAtStartPar
Tensor reducido al ‘slice’ indicado a lo largo de los ejes especificados.

\sphinxlineitem{Return type}
\sphinxAtStartPar
{\hyperref[\detokenize{myutils:myutils.tensor.Tensor}]{\sphinxcrossref{Tensor}}}

\end{description}\end{quote}

\end{fulllineitems}

\index{squeeze() (myutils.tensor.Tensor method)@\spxentry{squeeze()}\spxextra{myutils.tensor.Tensor method}}

\begin{fulllineitems}
\phantomsection\label{\detokenize{myutils:myutils.tensor.Tensor.squeeze}}
\pysigstartsignatures
\pysiglinewithargsret{\sphinxbfcode{\sphinxupquote{squeeze}}}{}{}
\pysigstopsignatures
\sphinxAtStartPar
Elimina todos los ejes del tensor cuya dimensión sea igual a ‘1’.
\begin{quote}\begin{description}
\sphinxlineitem{Returns}
\sphinxAtStartPar
Tensor resultante después de eliminar los ejes con dimensión ‘1’.

\sphinxlineitem{Return type}
\sphinxAtStartPar
{\hyperref[\detokenize{myutils:myutils.tensor.Tensor}]{\sphinxcrossref{Tensor}}}

\end{description}\end{quote}

\end{fulllineitems}

\index{std() (myutils.tensor.Tensor method)@\spxentry{std()}\spxextra{myutils.tensor.Tensor method}}

\begin{fulllineitems}
\phantomsection\label{\detokenize{myutils:myutils.tensor.Tensor.std}}
\pysigstartsignatures
\pysiglinewithargsret{\sphinxbfcode{\sphinxupquote{std}}}{\sphinxparam{\DUrole{n}{axe}}}{}
\pysigstopsignatures
\sphinxAtStartPar
Devuelve un Tensor comprimido en el eje ‘axe’, suprimiéndolo y calculando la desviación estándar a lo largo de dicho eje.
\begin{quote}\begin{description}
\sphinxlineitem{Parameters}
\sphinxAtStartPar
\sphinxstyleliteralstrong{\sphinxupquote{axe}} (\sphinxstyleliteralemphasis{\sphinxupquote{int}}) \textendash{} Índice del eje a comprimir y calcular la desviación estándar.

\sphinxlineitem{Returns}
\sphinxAtStartPar
Tensor con el eje ‘axe’ suprimido y cada punto contiene la desviación estándar a lo largo de ese eje.

\sphinxlineitem{Return type}
\sphinxAtStartPar
{\hyperref[\detokenize{myutils:myutils.tensor.Tensor}]{\sphinxcrossref{Tensor}}}

\end{description}\end{quote}

\end{fulllineitems}

\index{swap() (myutils.tensor.Tensor method)@\spxentry{swap()}\spxextra{myutils.tensor.Tensor method}}

\begin{fulllineitems}
\phantomsection\label{\detokenize{myutils:myutils.tensor.Tensor.swap}}
\pysigstartsignatures
\pysiglinewithargsret{\sphinxbfcode{\sphinxupquote{swap}}}{\sphinxparam{\DUrole{n}{axe1}}\sphinxparamcomma \sphinxparam{\DUrole{n}{axe2}}}{}
\pysigstopsignatures
\sphinxAtStartPar
Devuelve un Tensor con el mismo “array” pero los ejes “axe1” y “axe2” intercambiados.
\begin{quote}\begin{description}
\sphinxlineitem{Parameters}\begin{itemize}
\item {} 
\sphinxAtStartPar
\sphinxstyleliteralstrong{\sphinxupquote{axe1}} (\sphinxstyleliteralemphasis{\sphinxupquote{int}}) \textendash{} Índice del primer eje a intercambiar.

\item {} 
\sphinxAtStartPar
\sphinxstyleliteralstrong{\sphinxupquote{axe2}} (\sphinxstyleliteralemphasis{\sphinxupquote{int}}) \textendash{} Índice del segundo eje a intercambiar.

\end{itemize}

\sphinxlineitem{Returns}
\sphinxAtStartPar
Tensor con los ejes “axe1” y “axe2” intercambiados.

\sphinxlineitem{Return type}
\sphinxAtStartPar
{\hyperref[\detokenize{myutils:myutils.tensor.Tensor}]{\sphinxcrossref{Tensor}}}

\end{description}\end{quote}

\end{fulllineitems}

\index{take() (myutils.tensor.Tensor method)@\spxentry{take()}\spxextra{myutils.tensor.Tensor method}}

\begin{fulllineitems}
\phantomsection\label{\detokenize{myutils:myutils.tensor.Tensor.take}}
\pysigstartsignatures
\pysiglinewithargsret{\sphinxbfcode{\sphinxupquote{take}}}{\sphinxparam{\DUrole{o}{**}\DUrole{n}{axes}}}{}
\pysigstopsignatures
\sphinxAtStartPar
Toma una serie de ejes (por nombre), igualados a una lista de índices, y devuelve un Tensor
reducido a los índices seleccionados.
\begin{quote}\begin{description}
\sphinxlineitem{Parameters}
\sphinxAtStartPar
\sphinxstyleliteralstrong{\sphinxupquote{axes}} (\sphinxstyleliteralemphasis{\sphinxupquote{dict}}) \textendash{} Argumentos clave\sphinxhyphen{}valor donde las claves son los nombres de los ejes y los valores
son listas de índices a lo largo de esos ejes.

\sphinxlineitem{Returns}
\sphinxAtStartPar
Tensor reducido a los índices seleccionados a lo largo de los ejes especificados.

\sphinxlineitem{Return type}
\sphinxAtStartPar
{\hyperref[\detokenize{myutils:myutils.tensor.Tensor}]{\sphinxcrossref{Tensor}}}

\end{description}\end{quote}

\end{fulllineitems}


\end{fulllineitems}

\index{append() (in module myutils.tensor)@\spxentry{append()}\spxextra{in module myutils.tensor}}

\begin{fulllineitems}
\phantomsection\label{\detokenize{myutils:myutils.tensor.append}}
\pysigstartsignatures
\pysiglinewithargsret{\sphinxcode{\sphinxupquote{myutils.tensor.}}\sphinxbfcode{\sphinxupquote{append}}}{\sphinxparam{\DUrole{n}{axe}}\sphinxparamcomma \sphinxparam{\DUrole{o}{*}\DUrole{n}{args}}}{}
\pysigstopsignatures
\sphinxAtStartPar
Concatena dos objetos ‘Tensor’ a lo largo de un eje especificado por su nombre.
\begin{quote}\begin{description}
\sphinxlineitem{Parameters}\begin{itemize}
\item {} 
\sphinxAtStartPar
\sphinxstyleliteralstrong{\sphinxupquote{axe}} (\sphinxstyleliteralemphasis{\sphinxupquote{str}}) \textendash{} Nombre del eje a lo largo del cual se realizará la concatenación.

\item {} 
\sphinxAtStartPar
\sphinxstyleliteralstrong{\sphinxupquote{args}} ({\hyperref[\detokenize{myutils:myutils.tensor.Tensor}]{\sphinxcrossref{\sphinxstyleliteralemphasis{\sphinxupquote{Tensor}}}}}) \textendash{} Tensores que se concatenarán. Se deben proporcionar al menos dos tensores de entrada.
Todos los tensores deben tener un parámetro ‘Tensor.axis’ idéntico y las mismas
dimensiones en todos los ejes excepto posiblemente en ‘axe’.

\end{itemize}

\sphinxlineitem{Returns}
\sphinxAtStartPar
Nuevo Tensor que consiste en la concatenación de los argumentos en el orden proporcionado.

\sphinxlineitem{Return type}
\sphinxAtStartPar
{\hyperref[\detokenize{myutils:myutils.tensor.Tensor}]{\sphinxcrossref{Tensor}}}

\sphinxlineitem{Raises}
\sphinxAtStartPar
\sphinxstyleliteralstrong{\sphinxupquote{ValueError}} \textendash{} Si no se cumplen los requisitos de entrada (al menos dos tensores y ‘axe’).

\end{description}\end{quote}

\end{fulllineitems}

\index{load\_tensor() (in module myutils.tensor)@\spxentry{load\_tensor()}\spxextra{in module myutils.tensor}}

\begin{fulllineitems}
\phantomsection\label{\detokenize{myutils:myutils.tensor.load_tensor}}
\pysigstartsignatures
\pysiglinewithargsret{\sphinxcode{\sphinxupquote{myutils.tensor.}}\sphinxbfcode{\sphinxupquote{load\_tensor}}}{\sphinxparam{\DUrole{n}{file\_name}}}{}
\pysigstopsignatures
\sphinxAtStartPar
Devuelve un Tensor almacenado en un archivo de texto.
\begin{quote}\begin{description}
\sphinxlineitem{Parameters}
\sphinxAtStartPar
\sphinxstyleliteralstrong{\sphinxupquote{file\_name}} (\sphinxstyleliteralemphasis{\sphinxupquote{str}}) \textendash{} Nombre del archivo que contiene el Tensor.

\sphinxlineitem{Returns}
\sphinxAtStartPar
Tensor cargado desde el archivo de texto.

\sphinxlineitem{Return type}
\sphinxAtStartPar
{\hyperref[\detokenize{myutils:myutils.tensor.Tensor}]{\sphinxcrossref{Tensor}}}

\end{description}\end{quote}

\end{fulllineitems}



\chapter{Indices and tables}
\label{\detokenize{index:indices-and-tables}}\begin{itemize}
\item {} 
\sphinxAtStartPar
\DUrole{xref,std,std-ref}{genindex}

\item {} 
\sphinxAtStartPar
\DUrole{xref,std,std-ref}{modindex}

\item {} 
\sphinxAtStartPar
\DUrole{xref,std,std-ref}{search}

\end{itemize}


\renewcommand{\indexname}{Python Module Index}
\begin{sphinxtheindex}
\let\bigletter\sphinxstyleindexlettergroup
\bigletter{m}
\item\relax\sphinxstyleindexentry{myutils}\sphinxstyleindexpageref{myutils:\detokenize{module-myutils}}
\item\relax\sphinxstyleindexentry{myutils.bool}\sphinxstyleindexpageref{myutils:\detokenize{module-myutils.bool}}
\item\relax\sphinxstyleindexentry{myutils.stats}\sphinxstyleindexpageref{myutils:\detokenize{module-myutils.stats}}
\item\relax\sphinxstyleindexentry{myutils.tensor}\sphinxstyleindexpageref{myutils:\detokenize{module-myutils.tensor}}
\end{sphinxtheindex}

\renewcommand{\indexname}{Index}
\printindex
\end{document}